\documentclass[fleqn,11pt]{article}

\usepackage[letterpaper,margin=0.75in]{geometry}

\usepackage{amsmath}
\usepackage{booktabs}
\usepackage{graphicx}
\usepackage{listings}

\setlength{\parindent}{1.4em}

\begin{document}

\lstset{
  language=Python,
  basicstyle=\small,          % print whole listing small
  keywordstyle=\bfseries,
  identifierstyle=,           % nothing happens
  commentstyle=,              % white comments
  stringstyle=\ttfamily,      % typewriter type for strings
  showstringspaces=false,     % no special string spaces
  numbers=left,
  numberstyle=\tiny,
  numbersep=5pt,
  frame=tb,
}

\title{Lab 1 Report}

\author{Martin Sanchez and Ben Jacobson}

\date{2/1/17}

\maketitle

\section{Two Nodes}

In this first section we explored a simple network consisting of two nodes and one bidirectional link. We explored the following scenarios:
\begin{enumerate}

\item Set the bandwidth of the links to 1 Mbps, with a propagation delay of 1 second. Send one packet with 1000 bytes from n1 to n2 at time 0.

\item Set the bandwidth of the links to 100 bps, with a propagation delay of 10 ms. Send one packet witih 1000 bytes from n1 to n2 at time 0.

\item 3. Set the bandwidth of the links to 1 Mbps, with a propagation delay of 10 ms. Send three packets from n1 to n2 at time 0 seconds, then one packet at time 2 seconds. All packets should have 1000 bytes.

\end{enumerate}
Our findings for each of the above scenarios, include the following:
\begin{enumerate}

\item Our network configuration

\item The output of the simulation

\item The calculations we used in order to verify the output was correct.

\end{enumerate}

The results of running the simulator for each of the scenarios are below:

\begin{enumerate}

\item 

\item The output of the simulation

\item The calculations we used in order to verify the output was correct.

\end{enumerate}

% \begin{lstlisting}
% class Node:
%     def __init__(self,scheduler):
%         self.scheduler = scheduler

%     def handle_message(self,t,message):
%         print "Received at",t,':',message.body
%         if message.times < 3:
%             self.scheduler.add(t+1.5, message, self.handle_message)
%         message.times += 1
% \end{lstlisting}

\section{Three Nodes}

Mauris dictum augue a eros adipiscing mollis. Duis tempus, risus sed
iaculis vehicula, mauris mi aliquam odio, aliquet congue ligula tortor
vitae leo. In convallis, lectus sed egestas tincidunt, augue massa
lacinia augue, a ornare dui magna id enim. Fusce porttitor scelerisque
lorem nec eleifend. Cras lobortis eleifend orci, non lacinia felis
tincidunt eget. Nam vulputate tellus magna, at scelerisque
ligula. Duis dictum bibendum odio nec lobortis. Ut dignissim fringilla
euismod. In pharetra augue et odio blandit malesuada. Nulla lacus
nisi, auctor eget aliquet a, auctor at lorem. Suspendisse nec laoreet
sapien. Nulla facilisi. Nam a congue nunc. Pellentesque auctor turpis
ac augue aliquam convallis. Aenean sit amet eros nibh. Morbi a egestas
libero.

\vspace{0.5cm}
\begin{tabular}{lc}
  \toprule
  Setting & Result\\
  \midrule
  1 & 1.0\\
  2 & 3.45\\
  3 & 7.85\\
  4 & 15.89\\
  \bottomrule
\end{tabular}
\vspace{0.5cm}

Nam sed lacus sit amet nisl bibendum rutrum vel id nisl. Etiam sit
amet ipsum vulputate tellus fringilla tristique a et augue. Etiam
suscipit ante id est lobortis hendrerit. Vivamus vel nisl sit amet
metus volutpat faucibus. Praesent nunc urna, luctus vel convallis
eget, luctus et odio. Nunc et nisl felis. Fusce quis libero sit amet
libero cursus pretium. Vivamus dictum risus non tellus commodo non
bibendum tortor convallis. Cras tempor orci eu leo auctor sed euismod
arcu consectetur. In scelerisque felis et erat commodo
bibendum. Pellentesque hendrerit enim vitae neque sollicitudin
bibendum. In ligula lorem, blandit sit amet aliquet eget, accumsan ut
sem. Maecenas in velit justo. Morbi tellus sem, ultricies in tristique
non, aliquam a lacus. Sed rhoncus blandit ligula, ut eleifend magna
lacinia quis.

\begin{enumerate}

\item Item.

\item Another item.

\end{enumerate}

\section{Queueing Theory}



% \includegraphics[width=11cm]{graphs/download-combined}

\section{Section Name}

$d_{trans}$ is the transmission delay. $d_{prop}$ is the propagation delay.

\begin{align*}
d &= d_{trans} + d_{prop}\\
  &= (1000*8)/1000000 + 0.05\\
  &= 0.058
\end{align*}

\end{document}